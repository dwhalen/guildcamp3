%%%%%
%%
%% Notebooks live in this directory.  There are two kinds: in-game
%% notebooks (like classic research notebooks) and out-of-game
%% notebooks (also known as green notebooks).  Both kinds are defined
%% in Lists/notebook-LIST.tex.  In-game notebooks, defined as Notebook
%% macros, get assigned in \MYitems.  Out-of-game notebooks, defined
%% as GreenNotebook macros, get assigned in \MYgreens.
%%
%% This file doubles as a latex'able example out-of-game notebook.
%% README.tex, also in this directory, doubles as an example in-game
%% notebook.
%%
%% Notebook and GreenNotebook macros each have a file that lives here.
%% The argument to \startnotebook{...}  probably should be the macro
%% for the given notebook.  However, you can also just use
%% \startnotebook{Some Text} if you want.
%%
%% Note that every \startnotebook command needs a matching
%% \endnotebook command.  Also note that no ownership information
%% appears on the notebook.
%%
%%
%% If you want to use notebooks for memory packets, see related macros
%% in Lists/mem-LIST.tex (and in Charsheets/README.tex).  If you want
%% more programmatic control over the structure/linkage of notebooks,
%% like for a computer security mechanic, see LabelCover and LabelPage
%% in Lists/sign-LIST.tex.
%%
%%%%%

\documentclass[greennotebook]{guildcamp3} %% [notebook] or [greennotebook]
\begin{document}

\startnotebook{\nEscapePodTech{}}

% this belongs to scientist 2 and can hold two people
% parts needed: sheet of scrap metal (page 1), flashlight (page 4), circuit board (page two), graphite lube (page three), cog (page two), transistor (page six), niobium carbide (page 5) - use a screwdriver


%% The argument to \begin{page}{...} is a reference string.  You use
%% \nbref{...} in page text to refer to a given page.  These
%% references will be typeset as numbers, based on the order of the
%% pages.
\begin{page}{first}
	
Well, if \cSciOne{} is just going to try to take all the credit in this world you may as well go somewhere else where your abilities can be recognized. Being an excellent scientist, you're pretty sure you know how to build an escape pod. Probably even one big enough to hold two people. Not that you'd ever bring \cSciOne{} of course. Better not let \cSciOne{\them} know what you're trying to do. 

Follow the instructions in this notebook to build an escape pod. 

Well, your first job here \emph{is} after all to help save the world. Turn to the \nbref{techpageone} after getting the proposal to save the world signed. 

\end{page}



%% the optional argument [One Minute] means this page will be labeled and
%% referenced with a string instead of a number.  The numbering of other
%% pages will ignore this one.
\begin{page}[The proposal is finally done...]{techpageone}

Well, the proposal was useful at least. You now realize you'll need a sheet of \iScrapMetal{} to build the escape pod. In all likelihood some of the other parts needed for the machine are probably useful for an escape pod too. Once you've gotten the machine built you can turn to \nbref{techpagetwo}.

\end{page}

\begin{page}[And some of those things really were useful]{techpagetwo}
	
Well, those things were useful for building a machine. Guess a \iCircuitBoard{} and a \iCog{} are useful for an escape pod too. All this building has been pretty tiring. Before being able to figure out what else you might need for a machine you'll have to recharge your battery. When you get your $\omega$ back up to 4, you can continue to \nbref{techpagethree}
	
\end{page}

\begin{page}[that's better...]{techpagethree}
	
Much better. Now if you can just make sure your escape pod runs smoothly. How would you do that? Oh yeah, \iGraphiteLube{} is pretty useful - make sure you have a tube of that for your escape pod. 

What else might you need to do build a machine? You find some scribbling in your notebook, but unfortunately the convergence seems to have scrambled it into something indecipherable: $\eta \nu \alpha \nu \omega \phi \epsilon \gamma \mu \eta$. Luckily there seems to be some sort of key provided, but you'll still have to unscramble it. When you've unscrambled it you can turn to \nbref{techpagefour}

\begin{tabular}{|c|c|c|c|c|c|c|c|}
	\hline \rule[-2ex]{0pt}{5.5ex} a & f & g & h & i & l & s & t \\ 
	\hline \rule[-2ex]{0pt}{5.5ex} $\alpha$ & $\phi$ & $\gamma$ & $\eta$ & $\epsilon$ & $\nu$ & $\mu$ & $\omega$ \\ 
	\hline 
\end{tabular} 	
\end{page}

\begin{page}[silly convergence...]{techpagefour}
	
Well, that's good - a \iFlashlight{} will be good to have in an escape pod. Headlights are nice. Would be good to talk to someone after all that work. Have a five-minute conversation with a magical person and you can turn to \nbref{techpagefive}
	
\end{page}

\begin{page}[what a refreshing conversation...]{techpagefive}

Well, that conversation cleared your head. Now you remember that you're going to need some \iNiobiumCarbide{} to build the pod. Guess it'll stop any unsuspecting aliens from breaking in. Too bad all this distraction has lowered your $\omega$ by one. 

You know, maybe some gold will be helpful once you arrive in the alternate universe. Acquire 10 gold to open \nbref{techpagesix}
	
\end{page}

\begin{page}[hope they like money...]{techpagesix}
	
Now you've got some money. Probably ought to be able to buy yourself the last thing needed to complete the escape pod now that you remember it's an \iTransistor{}. 

Well, now you've got to build the machine. You'll need to gather all the pieces needed and use a \iScrewdriver{}. You must work on them with the screwdriver for five uninterrupted minutes. Once you've done so, you may turn to page \nbref{last}. 

As a reminder, the items needed were: \iScrapMetal{}, \iCircuitBoard{}, \iCog{}, \iGraphiteLube{}, \iFlashlight{}, \iNiobiumCarbide{}, and \iTransistor{}. 
	
\end{page}	


\begin{page}[at last...]{last}
	
Destroy the items needed for building the escape pod other than the \iScrewdriver{}.

Your escape pod is created. It's a small pod, but big enough for you and one friend. If you want to escape to a random third universe, then sign each of your names below and ensure you have at least three units of some combination of $\alpha$ and $\omega$ at game time T+4:00. 

Passenger 1:............................................................

Passenger 2:............................................................


\end{page}



\endnotebook

\end{document}
