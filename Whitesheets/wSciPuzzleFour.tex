%%%%%
%%
%% Whitesheets live in this directory.  This file doubles as a
%% latex'able example whitesheet.
%%
%% _template.tex serves as a bare-bones template suitable for
%% copying when starting a new sheet.
%%
%% Whitesheet macros (in ../Lists/white-LIST.tex, presumably) each
%% have a file that lives here.  The argument to \name{...} probably
%% should be the macro for the given whitesheet.  However, you can also
%% just use \name{Some Text} if you want.
%%
%% Note that for whitesheets, \name doesn't actualy typeset anything.
%% If you want the ``title'' of the IG document to appear, typeset it
%% how you want it.  Similarly, no ownership information appears on
%% the sheet.
%%
%% If the whitesheet has a number (via \MYnumber in
%% Lists/white-LIST.tex), \name{<white macro>} will cause the number
%% to printed as an out-of-game note.
%%
%%%%%

\documentclass[white]{guildcamp3}
\begin{document}

\name{\wSciPuzzleFour{}} %% used as a label, doesn't typeset anything
\large\textbf{Scientist Puzzle for Component Four}  


Solve this to determine the item needed for component four. You don't need very many of this item. The number you need is the multiplicative identity. 

\oog{These puzzles are often known as Hanjie or Nonograms. Your aim in these puzzles is to colour the whole grid in to black and white squares. At the top of each column, and at the side of each row, you will notice a set of one or more numbers. These numbers tell you the runs of black squares in that row/column. So, if you see '10 1', that tells you that there will be a run of exactly 10 black squares, followed by one or more white square, followed by a single black square. There may be more white squares before/after this sequence.}

\small 

\begin{tabular}{|c|c|c|c|c|c|c|c|c|c|c|c|c|c|c|c|c|c|c|c|c|c|c|c|c|c|c|c|c|}
	\hline \rule[-2ex]{0pt}{5.5ex}  &  &  &  &  &  &  &  &  &  &  &  &  &  &  &  &  &  &  &  &  &  &  &  &  & 2 & 2 &  &  \\ 
	   &  &  & 3 & 2 & 2 & 2 & 1 &  &  &  & 3 & 2 & 2 & 2 & 2 & 4 &  &  &  &  &  & 3 & 2 & 2 & 1 & 1 & 2 & 1 \\ 
	   & 7 & 9 & 2 & 2 & 2 & 1 & 2 & - & 3 & 8 & 4 & 3 & 2 & 2 & 1 & 3 & 8 & 5 & - & 5 & 7 & 3 & 2 & 2 & 2 & 2 & 5 & 5 \\ 
	\hline \rule[-2ex]{0pt}{5.5ex} 4, 5, 5 &  &  &  &  &  &  &  &  &  &  &  &  &  &  &  &  &  &  &  &  &  &  &  &  &  &  &  &  \\ 
	\hline \rule[-2ex]{0pt}{5.5ex} 6, 8, 7 &  &  &  &  &  &  &  &  &  &  &  &  &  &  &  &  &  &  &  &  &  &  &  &  &  &  &  &  \\ 
	\hline \rule[-2ex]{0pt}{5.5ex} 3, 2, 2, 2 &  &  &  &  &  &  &  &  &  &  &  &  &  &  &  &  &  &  &  &  &  &  &  &  &  &  &  &  \\ 
	\hline \rule[-2ex]{0pt}{5.5ex} 2, 2, 3, 3 &  &  &  &  &  &  &  &  &  &  &  &  &  &  &  &  &  &  &  &  &  &  &  &  &  &  &  &  \\ 
	\hline \rule[-2ex]{0pt}{5.5ex} 2, 2, 2, 2 &  &  &  &  &  &  &  &  &  &  &  &  &  &  &  &  &  &  &  &  &  &  &  &  &  &  &  &  \\ 
	\hline \rule[-2ex]{0pt}{5.5ex} 2, 2, 2, 2, 4 &  &  &  &  &  &  &  &  &  &  &  &  &  &  &  &  &  &  &  &  &  &  &  &  &  &  &  &  \\ 
	\hline \rule[-2ex]{0pt}{5.5ex} 2, 3, 2, 2, 2 &  &  &  &  &  &  &  &  &  &  &  &  &  &  &  &  &  &  &  &  &  &  &  &  &  &  &  &  \\ 
	\hline \rule[-2ex]{0pt}{5.5ex} 2, 2, 3, 3, 2 &  &  &  &  &  &  &  &  &  &  &  &  &  &  &  &  &  &  &  &  &  &  &  &  &  &  &  &  \\ 
	\hline \rule[-2ex]{0pt}{5.5ex} 3, 1, 3, 2, 3, 2 &  &  &  &  &  &  &  &  &  &  &  &  &  &  &  &  &  &  &  &  &  &  &  &  &  &  &  &  \\ 
	\hline \rule[-2ex]{0pt}{5.5ex} 6, 6, 7 &  &  &  &  &  &  &  &  &  &  &  &  &  &  &  &  &  &  &  &  &  &  &  &  &  &  &  &  \\ 
	\hline \rule[-2ex]{0pt}{5.5ex} 2, 3, 3 &  &  &  &  &  &  &  &  &  &  &  &  &  &  &  &  &  &  &  &  &  &  &  &  &  &  &  &  \\ 
	\hline 
\end{tabular} 
\end{document}
