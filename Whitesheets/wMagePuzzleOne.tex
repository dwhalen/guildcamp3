%%%%%
%%
%% Whitesheets live in this directory.  This file doubles as a
%% latex'able example whitesheet.
%%
%% _template.tex serves as a bare-bones template suitable for
%% copying when starting a new sheet.
%%
%% Whitesheet macros (in ../Lists/white-LIST.tex, presumably) each
%% have a file that lives here.  The argument to \name{...} probably
%% should be the macro for the given whitesheet.  However, you can also
%% just use \name{Some Text} if you want.
%%
%% Note that for whitesheets, \name doesn't actualy typeset anything.
%% If you want the ``title'' of the IG document to appear, typeset it
%% how you want it.  Similarly, no ownership information appears on
%% the sheet.
%%
%% If the whitesheet has a number (via \MYnumber in
%% Lists/white-LIST.tex), \name{<white macro>} will cause the number
%% to printed as an out-of-game note.
%%
%%%%%

\documentclass[white]{guildcamp3}
\begin{document}

\name{\wMagePuzzleOne{}} %% used as a label, doesn't typeset anything

\large\textbf{Mage Puzzle for Ingredient One}  

Solve this to determine the number you need of each ingredient. 

\oog{These puzzles are often known as Hanjie or Nonograms. Fill in each square as black or white. The numbers on the side and top indicate the sizes of the blocks of black in that row or column. Two blocks in a row/column will not touch each other.}

\begin{tabular}{|c|c|c|c|c|c|c|c|}
	\hline \rule[-2ex]{0pt}{5.5ex}  &  &  &  & 2 & 2 &  &  \\ 
	  & 1 & 2 & 2 & 2 & 2 & 5 & 4 \\ 
      & 2 & 3 & 4 & 2 & 2 & 2 & 2 \\ 
	\hline \rule[-2ex]{0pt}{5.5ex} 3 &  &  &  &  &  &  &  \\ 
	\hline \rule[-2ex]{0pt}{5.5ex} 6 &  &  &  &  &  &  &  \\ 
	\hline \rule[-2ex]{0pt}{5.5ex} 2 2 &  &  &  &  &  &  &  \\ 
	\hline \rule[-2ex]{0pt}{5.5ex} 2 &  &  &  &  &  &  &  \\ 
	\hline \rule[-2ex]{0pt}{5.5ex} 2 &  &  &  &  &  &  &  \\ 
	\hline \rule[-2ex]{0pt}{5.5ex} 2 &  &  &  &  &  &  &  \\ 
	\hline \rule[-2ex]{0pt}{5.5ex} 2 &  &  &  &  &  &  &  \\ 
	\hline \rule[-2ex]{0pt}{5.5ex} 2 &  &  &  &  &  &  &  \\ 
	\hline \rule[-2ex]{0pt}{5.5ex} 2 &  &  &  &  &  &  &  \\ 
	\hline \rule[-2ex]{0pt}{5.5ex} 7 &  &  &  &  &  &  &  \\ 
	\hline \rule[-2ex]{0pt}{5.5ex} 7 &  &  &  &  &  &  &  \\ 
	\hline 
\end{tabular} 

What ingredient is it? Fill in the middle word such that the pairing with each outer word makes sense in each of these cases (the full three words do not have to make sense).

\begin{tabular}{|c|c|c|}
	\hline \rule[-2ex]{0pt}{5.5ex} little & ... ... ... ... ... & sticks \\ 
	\hline \rule[-2ex]{0pt}{5.5ex} barbeque chicken & ... ... ... ... ... & of fire \\ 
	\hline 
\end{tabular} 

\end{document}
