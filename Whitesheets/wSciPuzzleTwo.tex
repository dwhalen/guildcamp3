%%%%%
%%
%% Whitesheets live in this directory.  This file doubles as a
%% latex'able example whitesheet.
%%
%% _template.tex serves as a bare-bones template suitable for
%% copying when starting a new sheet.
%%
%% Whitesheet macros (in ../Lists/white-LIST.tex, presumably) each
%% have a file that lives here.  The argument to \name{...} probably
%% should be the macro for the given whitesheet.  However, you can also
%% just use \name{Some Text} if you want.
%%
%% Note that for whitesheets, \name doesn't actualy typeset anything.
%% If you want the ``title'' of the IG document to appear, typeset it
%% how you want it.  Similarly, no ownership information appears on
%% the sheet.
%%
%% If the whitesheet has a number (via \MYnumber in
%% Lists/white-LIST.tex), \name{<white macro>} will cause the number
%% to printed as an out-of-game note.
%%
%%%%%

\documentclass[white]{guildcamp3}
\begin{document}

\name{\wSciPuzzleTwo{}} %% used as a label, doesn't typeset anything

\large\textbf{Scientist Puzzle for Component Two}  

Rearrange the following pairs of letters to form a component you need for the final machine:

\begin{tabular}{|c|c|c|c|c|}
	\hline \rule[-2ex]{0pt}{5.5ex} S & ST & SI & RE & OR \\ 
	\hline 
\end{tabular} 

The number of units of these you need is equal to the numerical part of the solution to the following:

If you have two 100 microFarad capacitors in series, how much total capacitance would you have?

Ans: .. microFarad

\oog{Hint: you add capacitors in series the same way you add resistors in parallel.}

Once you know the first answer you'll be able to figure out the units for this one. 

\end{document}
