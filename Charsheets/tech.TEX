%%%%%
%%
%% Character sheets live in this directory.  This file doubles as a
%% latex'able example charsheet.
%%
%% _template.tex serves as a bare-bones template suitable for
%% copying when starting a new sheet.
%%
%% Character macros (in ../Lists/char-LIST.tex, presumably) each
%% have a file that lives here.  The argument to \name{...} probably
%% should be the macro for the given character, which will generate
%% the charsheet's name (and print out lists of the characters stuff
%% at the end) as specified in char-LIST.tex.  However, you can also
%% just use \name{Some Text} if you want.
%%
%%%%%

\documentclass[char]{guildcamp3}
\begin{document}

\name{\cTech{}}

%% This sort of use of \updatemacro is covered in
%% Extras/README-namemappings.
\updatemacro{\cNPC}{
  \unknownplayer %% doesn't know what he looks like
  }

%% quote examples
\bigquote{``Use this macro for large quotes of prose and such.  It
justifies everything like a paragraph, except with no
indentation.''}{-- The Author}

\cenquote{``This macro is good\\ For shorter quotes\\ Or things like
song lyrics:\\ It centers.''}{-- The Author}


%% \TODO outputs to the page and the terminal.  It is used for
%% reminders of future work, and a convenient way to build a short
%% outline for a sheet in-progress.
\TODO{Write this sheet.}

Officially, you've been brought along to this meeting to recharge batteries and provide maintenance for the scientists equipment. If nothing else, they do recognize your skill here - no doubt you're faster and better at recharging the batteries than anyone else around. They certainly break a lot less often and work better when you do it. Finally you convinced the higher ups (and in particular the Undersecretary of Technology, \cPoliTwo{\intro}) that you were worth bringing along. But you're certain you're capable of doing more than just charging the batteries  - while that's all you get credit for, it certainly isn't the only thing you plan to do. 

Unfortunately, there's two main scientists here as well though - \cSciOne{\intro} and \cSciTwo{\intro}. They're supposed to be your bosses. In charge of you - doesn't matter that you're older than them and have proved your worth many times over. They've got the fancy degrees, and you're just an old man who happens to be good at stuff. This whole hierarchy is silly - no reason you shouldn't have the most competent person be the one able to do the most useful tasks. 

%%%%%
%% The itemz environment is a list environment similar to itemize.
%% The typesetting is very tight, and matches that used by the lists
%% at the end of character sheets.  It takes an optional argument that
%% acts as a title for the list.  The enum environment is a similar
%% variation of the enumerate environment, and the desc environment is
%% similar to description.
\begin{itemz}[Goals]
  \item Make sure the batteries needed by the scientists remain charged. That is your official job after all - no reason to upset the Undersecretary of Technology by not doing so. 
  \item Finish your own research. You've been working on it for years and need to finish it before you all die. Who knows, maybe it will even give the insight to save you all?
  \item Spread the word of pacifism. Even if the world is ending, we all ought to respect each other.
  \item Find religious comfort. Get to a sermon or two. It will help you deal with all this frustration.
  \item Ensure the system for recognizing people for their actions is fair. Giving people positions just because they've got degrees, money or age is silly. Judgment should be based off your competence and performance. 
\end{itemz}

\begin{itemz}[Notes]
  \item You were born in London.
  \item You went to MIT, and never left.
\end{itemz}


%%%%%
%% List contacts, using \contact{<char macro>}
\begin{contacts}
  \contact{\cSciOne{}} The head scientist on this mission. Officially. 
  \contact{\cSciTwo{}} The subordinate scientist on this mission. Officially. 
  \contact{\cServant{}} Your double
\end{contacts}


%%%%%
%% \starttag{<tag>} <elements> \endtag 
%% Valid <tag> values are blues, greens, abils, combat, mems, items,
%% whites, notebooks, cash, signs, ids.  These each correspond to a
%% type of macro defined in Lists/.
%%
%% By using \starttag, you can give this character <elements> of the
%% type corresponding to <tag>.
%%
%% Multiple uses of the same <tag> will simply add together.
\starttag{mems}

  \mTest{}

  \memfold{``Rosebud''}{Rosebud!  That was the name of\ldots the name
  of\ldots darn, you forget.}

  \startmembook{Book of Mempackets}

    \mempage{if you see something blue}{Hey, that's blue!  Oh, you
    remember, blue is your favorite color.  You really like blue
    things, especially blue tentacles.  You wonder why\ldots}

    \mempage{``Octy''}{Octy!  You remember Octy now!  She was your pet
    blue octopus when you were a young child living offshore.  Oh, the
    fun times you had!

    You used to go swimming and diving with Octy all the time.  This
    was years ago.  What happened?  You still can't remember\ldots but
    you know you haven't even thought of her since you were small.}

  \endmembook

\endtag

\starttag{abils}
  \ability{Amazing Powers}{You can do strange and amazing things.}{I
  do something strange and amazing.}
\endtag


\end{document}
