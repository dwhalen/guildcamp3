%%%%%
%%
%% Character sheets live in this directory.  This file doubles as a
%% latex'able example charsheet.
%%
%% _template.tex serves as a bare-bones template suitable for
%% copying when starting a new sheet.
%%
%% Character macros (in ../Lists/char-LIST.tex, presumably) each
%% have a file that lives here.  The argument to \name{...} probably
%% should be the macro for the given character, which will generate
%% the charsheet's name (and print out lists of the characters stuff
%% at the end) as specified in char-LIST.tex.  However, you can also
%% just use \name{Some Text} if you want.
%%
%%%%%

\documentclass[char]{guildcamp3}
\begin{document}

\name{\cMageOne{}}

%% This sort of use of \updatemacro is covered in
%% Extras/README-namemappings.
\updatemacro{\cNPC}{
  \unknownplayer %% doesn't know what he looks like
  }

%% quote examples
%\bigquote{``Use this macro for large quotes of prose and such.  It
%justifies everything like a paragraph, except with no
%indentation.''}{-- The Author}

%\cenquote{``This macro is good\\ For shorter quotes\\ Or things like
%song lyrics:\\ It centers.''}{-- The Author}


%% \TODO outputs to the page and the terminal.  It is used for
%% reminders of future work, and a convenient way to build a short
%% outline for a sheet in-progress.
%\TODO{This is a test character sheet.}

%% Using a Char macro with an empty argument, like \cTest{}, will
%% produce the \usual namemapping (see Extras/README-namemappings).
%% Introduce a character with \intro, e.g. \cTest{\intro}, to get a
%% more full name.  \intro can be used whenever it fits into the text
%% flow.
%Your friend from out of town, \cSomeGuy{\intro}, called you up one
%morning about a week ago.  \cSomeGuy{} told you to meet up with this
%person named \cNPC{}, whom you've never met.  \cNPC{}, whose full name
%is \cNPC{\intro}, is supposed to be really awesome and have a package
%for you.

%% You can skip the advanced namemapping commands, and instead use
%% \full, \fullplain (full name without prefixes or suffixes),
%% \formal, and \informal.  You can also nest these inside identities,
%% such as \nick{} (see Extras/README-identity).
%That bears repeating: \cNPC{\nick{\informal}}, whose full name is
%\cNPC{\full}, is supposed to be really awesome and have a package for
%you.

%% For pronouns and other gender-dependent words, you can use the
%% pronoun commands defined in Lists/char-LIST.tex to automatically
%% control them based on the character's gender.  For example,
%% \cTest{\They} will produce He, She, It, or He/She, based on
%% \cTest's \MYsex field.  You can define your own pronouns in
%% Lists/char-LIST.tex, as well.
%\cSomeGuy{} is a pretty good friend.  \cSomeGuy{\They} used to be your
%college roommate.  When \cSomeGuy{\they} called you up, you were
%pleasantly surprised to hear from \cSomeGuy{\them}.

%% \me{} produces whatever the argument to \name{...} is.  If, like
%% usual, the argument was a char Macro, \me{} is an alias for that
%% macro.
%Lots of people think your name, \me{\intro}, is funny.  You're not
%sure why; you think it's a fine name.

You have spent many long years studying the arcane force of your world. Your family supported your studies with a cold distance. You are a third born, not expected to take on the family's noble mantle ever, you are not trained in the intricacies of the noble etiquette instead being not so subtly directed to study either the high arcane or the high divine. In your case you dove into the lore of the magical to the exclusion of all else. Your time pouring over books and scrolls have left you with a lack of social grace and a lack of any friends with whom to share your discoveries, your times happy and sad. 

Still you are master of your craft with few that can rival your grasp of the weave of reality the sings to you. Your excellence was noticed by the research department of \bMagicWorld{} and you were brought in as a researcher that quickly rose through the ranks due to pure skill alone, no nepotism or politicking involved to be sure. Even as you rose there has been someone always there, \cMageTwo{\intro}. You have known \cMageTwo{\They} since you both began your arcane training, and since then there was a sense of competition. Overtime this sense has grown, with their amazing grasp of the weave being increasingly frustrating. You know that you are the better mage despite \cMageTwo{\They} being there with \cMageTwo{\Their} own advances and discoveries. You want to show everyone that you, you alone, are the best, not \cMageTwo{\They}. The best way to do this is to get your name on the department publishing board used to record the achievements of the mages. 

Recently you have come to know a dark secret that is befalling your world, the weave is becoming unstable which would lead to the end of your world. While working here you have met someone who has, dare you even think it, stolen your heart, \cNobleOne{\intro}. one of your superiors who you have barely gotten to know, but who has nonetheless left to shivering with nerves, and with the world perhaps coming to a close you have decided to put your hesitation to and end and craft a proposal ring and confess your love of \cNobleOne{}. Thankfully your department has been able to detect a stabilizing force in another place, a place outside this world, outside of conventional travel. While many have thought of alternate universes being out there, you have helped to create a method of taking that theory and making it a reality. There are somethings on the otherside that should be of use for stabilizing the weave, at least that is what your rituals have been able to divine. Small transient portals have been able to be opened, enough to see that this other world is not inhospitable, the air is breathable, though you have seen glimpses of the oddest sort of buildings, a shiny grey and glass. You wonder what forms of magic there are in this land, perhaps different techniques are used on the other side, so exciting this is. Your rituals have narrowed down the location of the artifacts you need, and your overseers have decided that a massive form of the ritual must be enacted. Under the watch of your superiors \cNobleOne{\intro} and \cNobleTwo{\intro}, as well as base security, the ritual was put into motion, with the rather unwanted help of \cMageTwo{}. There was a flash a sense of wrongness and a taste of red, there are a number of other people in strange attire, one looks so strangely like your twin, and the walls, they are no longer just the simple stone you are used to but a mix of stone and metal, and your ritual station now has an odd metal mirror of itself on the other side of the room. The ritual seems to have taken the your base and blended it with a base on the otherside. The world is one parallel to yours to the point that there are even doubles of yourselves there. It seems their side mirrors your situation and their attempted to come to your side seeking three artifacts that you know are on base somewhere. You and your rival will need to work with these newcomers to create something to analyze the artifacts in question to see how they can stabilize the worlds. Of course any plans will need to be signed off on by \cNobleTwo{} before any building is allowed, bureaucracy at its finest. Also fortunate is that while the rest of bothe worlds are cut off at the moment \cPaladin{} is here, meaning your mana will be able to be recharged, thank the gods. As you see your double and think to yourself that maybe they will be someone you can finally relate to, maybe even actually call friend, after all who would be a better friend than yourself. 

%%%%%
%% The itemz environment is a list environment similar to itemize.
%% The typesetting is very tight, and matches that used by the lists
%% at the end of character sheets.  It takes an optional argument that
%% acts as a title for the list.  The enum environment is a similar
%% variation of the enumerate environment, and the desc environment is
%% similar to description.
\begin{itemz}[Goals]
  \item Figure out how to save the world
  \item Win the research race against \cMageTwo{}
  \item propose funding and get signed off by \cNobleTwo{}
  \item Make a promise ring for \cNobleOne{} 
  \item Befriend \cSciTwo{}
  \item look for evidence of magic in the tech world
  \item Create useful items with your arcane knowledge
\end{itemz}

\begin{itemz}[Notes]
  \item You were born in London.
  \item You went to MIT, and never left.
\end{itemz}


%%%%%
%% List contacts, using \contact{<char macro>}
\begin{contacts}
  \contact{\cMageTwo{}} A child rival that is still around.
  \contact{\cNobleOne{}} The Secretariat of Defensive Measures, the ranking noble on the base
  \contact{\cNobleTwo{}} The Junior Secretariat of Arcane Research
  \contact{\cPaladin{}} The Paladin for the base, very well respected and source of mana 
  \contact{\cRogueOne{}} One of the security personnel for the base
  \contact{\cRogueTwo{}} One of the security personnel for the base
  \contact{\cServant{}} The loyal servant of /cNobleOne{} 
\end{contacts}

\end{document}
