\documentclass[char]{guildcamp3}
\begin{document}

\name{\cSpecOpOne{}}

%% This sort of use of \updatemacro is covered in
%% Extras/README-namemappings.
\updatemacro{\cNPC}{
  \unknownplayer %% doesn't know what he looks like
  }

%% quote examples
\bigquote{``Parallel universes seemed silly. Until you found yourself caught up in one and realized your double was up to no good.''}{-- The Author}

\cenquote{``Two worlds\\ two people \\ apparently the same \\ but definitely not''}{-- The Author}

% note: being the 3rd child is important
%% \TODO outputs to the page and the terminal.  It is used for
%% reminders of future work, and a convenient way to build a short
%% outline for a sheet in-progress.

Two worlds, that's funny. You never believed in the whole 'parallel universe' thing. Though now that it's happened it *is* kind of cool. Shame it had to happen under such inauspicious circumstances with the end of the world and all. Even more unfortunately, it seems that your double, \cRogueTwo{\intro}, isn't exactly the most upstanding citizen of the Magical World. Quite to the contrary, seems he's up to absolutely no good and constantly stealing stuff. You saw him try to grab a handful of gold from \cNobleTwo{\intro}! His commander! Of all people. 

Now, you've had your share of misgivings in your life. You were never the brightest in school - always behind the other kids and when it came time to apply to college, you never really got in. Of course many families would have just paid for you to get mental enhancement as a toddler and you would have caught right up. Unfortunately, your parents didn't believe in that and so you had to grow up as a mental inferior. Of course it didn't help that you were the third child so they had gotten tired of this stuff by the time they were born. Not getting into college gave you plenty of time to ponder this unfair predicament and you decided to take matters into your own hands and get the enhancement at that point. Too bad it isn't exactly offered to adults... To achieve your dreams you spent years finding the connections and money to finally get the surgery. You said you'd do anything and anything you did. It started off with small jobs - sweeping the floor at the local blood bank. But it escalated - you began engaging in petty theft - stealing lab equipment to let the underground doctors be able to perform the preliminary stages of the surgery. After that, highway robbery and grand theft for the money to pay for the next stage of the surgery. At some point in this whole process you even got so caught up in the ring of underground surgeons that you might have even let yourself be hired out as a hitman... This dark period of your life went on for a good 10 years or so.

Fortunately, around your 30th birthday you realized this wasn't what you wanted out of life. You'd had the mental enhancements for a good five years by then, and figured you ought to be able to figure out what to do with them. And whatever it was, it certainly wasn't this life of crime. OK, maybe getting arrested played into your decision here... Anyway, they decided to let you out of prison if you agreed to take a job with the National Special Operations Task Force. On the surface, you guys were supposed to be protecting the governmental figures, keeping the borders safe - stuff like that. Underneath, well there were plenty of more interesting operations. Unfortunately, it seems \cRogueTwo{\formal} never had a turning point in \cRogueTwo{\their} life - he's still up to no good. And if \cRogueTwo{they} are really your double, there is no limit to what \cRogueTwo{\they} could do. And no predicting it either. Neutralizing \cRogueTwo{\their} behavior is your only option. 

A couple years after you started working there, your child, \cSciOne{\intro}, was born. \cSciOne{\they} was a perfect baby and even somewhat bright. Regardless, when \cSciOne{\they} reached the age of three you scheduled \cSciOne{\them} for mental enhancement. There was no way you would let any child of yours go through what you did. You would do anything to put them at an advantage - read them books every night, play classical music - anything. As \cSciOne{\informal} grew up, you gave \cSciOne{\them} everything \cSciOne{\they} could need - as many books, tutors, sports coaches and everything else necessary to ensure success. And it worked - \cSciOne{\informal} got into college where they studied science and graduated at the top of their class. With your help of course - \cSciOne{\they} were never lacking in tutors, equipment, books or money. Eventually \cSciOne{} came to work for the government as well, as the lead research scientist on this parallel worlds thing. Both of you are here today to help figure out how to save the Technical World from annihilation. Whatever it takes to save the world, you're sure \cSciOne{\informal} can do it - and you'll provide whatever help you can. Of course, if this whole parallel universe thing is completely true who knows what atrocities \cRogueTwo{} might have exposed \cRogueTwo{their} child to. If that child is here today, make sure they know someone in the universe cares about them - any parallel of \cSciOne{\informal} deserves the the best. 

Now, officially the reason you're here today is as security. Being that \cSciOne{\informal} and \cSciOne{\their} colleague were certain about the nearing end of the world, a party has been organized to try to save it in its final few hours. Amongst the members is your boss, the Secretary of Defense -\cPoliOne{\intro}, and \cSciOne{\informal}'s boss, the Undersecretary of Technology - \cPoliTwo{\intro}. It is your official job today to ensure their safety. And now that classified base you work on has been mashed up with the equivalent classified base in the Magical World's, to figure out what happened and inventory the remaining supplies to ensure there are sufficient for the final plans. Critically, you need to keep the artifacts safe - the politicians and \cSciOne{\informal} agree that those are critical. 

You also have a new officer here with you today, \cSpecOpTwo{\intro}. Overall, \cSpecOpTwo{\they} seem to be a pretty good person, though perhaps a bit reticent. Make sure to keep \cSpecOpTwo{\them} busy and try to make sure they don't get too uncomfortable with the situation. Not that anyone is really comfortable with the end of the world, but do the best you can. 

%% You can skip the advanced namemapping commands, and instead use
%% \full, \fullplain (full name without prefixes or suffixes),
%% \formal, and \informal.  You can also nest these inside identities,
%% such as \nick{} (see Extras/README-identity).

%% For pronouns and other gender-dependent words, you can use the
%% pronoun commands defined in Lists/char-LIST.tex to automatically
%% control them based on the character's gender.  For example,
%% \cTest{\They} will produce He, She, It, or He/She, based on
%% \cTest's \MYsex field.  You can define your own pronouns in
%% Lists/char-LIST.tex, as well.


%%%%%
%% The itemz environment is a list environment similar to itemize.
%% The typesetting is very tight, and matches that used by the lists
%% at the end of character sheets.  It takes an optional argument that
%% acts as a title for the list.  The enum environment is a similar
%% variation of the enumerate environment, and the desc environment is
%% similar to description.
\begin{itemz}[Goals]
  \item Inventory the remaining supplies and ensure the resources needed to save the world are present.
  \item Keep \cSciOne{\informal{}} safe and ensure he has the resources needed to figure out how to save the Technical World. 
  \item Ensure the safety of the politicians, and the technical world in general.
  \item Make sure \cRogueTwo{\full{}} doesn't make a mess of things. 
\end{itemz}

\begin{itemz}[Notes]
  \item You were born in London.
  \item You went to MIT, and never left.
\end{itemz}


%%%%%
%% List contacts, using \contact{<char macro>}
\begin{contacts}
  \contact{\cSciOne{}} Your child. You would do anything to help \cSciOne{\them}. 
  \contact{\cRogueTwo{}} Your double. You are sure \cRogueTwo{\they} are up to no good. 
  \contact{\cPoliOne{}} The Secretary of Defense. Keep \cPoliOne{\them} safe and follow their orders.
  \contact{\cPoliTwo{}} The undersecretary of technology. Keep \cPoliTwo{\them} safe.
  \contact{\cSpecOpTwo{}} Your subordinate - a bit reticent.
  \contact{\cSciTwo{}} The other scientist here today. Make sure they don't do anything that might harm \cSciOne{\informal}.
\end{contacts}


%%%%%
%% \starttag{<tag>} <elements> \endtag 
%% Valid <tag> values are blues, greens, abils, combat, mems, items,
%% whites, notebooks, cash, signs, ids.  These each correspond to a
%% type of macro defined in Lists/.
%%
%% By using \starttag, you can give this character <elements> of the
%% type corresponding to <tag>.
%%
%% Multiple uses of the same <tag> will simply add together.
\starttag{mems}

  \mTest{}

  \memfold{``Rosebud''}{Rosebud!  That was the name of\ldots the name
  of\ldots darn, you forget.}

  \startmembook{Book of Mempackets}

    \mempage{if you see something blue}{Hey, that's blue!  Oh, you
    remember, blue is your favorite color.  You really like blue
    things, especially blue tentacles.  You wonder why\ldots}

    \mempage{``Octy''}{Octy!  You remember Octy now!  She was your pet
    blue octopus when you were a young child living offshore.  Oh, the
    fun times you had!

    You used to go swimming and diving with Octy all the time.  This
    was years ago.  What happened?  You still can't remember\ldots but
    you know you haven't even thought of her since you were small.}

  \endmembook

\endtag

\starttag{abils}
  \ability{Amazing Powers}{You can do strange and amazing things.}{I
  do something strange and amazing.}
\endtag


\end{document}
