\documentclass[char]{guildcamp3}
\begin{document}

\name{\cSciTwo{}}

\bigquote{``There is no great genius without a mixture of madness.''}{-- Aristotle}

It's not narcissism, just objective fact. I'm the smartest person I've ever met.

I could read sentences at age 1. I entered university at age 15. It's only been ten years since then, but I've been rising speedily through the scientific ranks in this science-focused society. I adore my work! And though I've yet to reach a major breakthrough, I am encouraged and mentored by some of the top members of my profession.

By the scientists, at any rate. The politicians who deal with the scientific world never fail to make me fume. They dominate science by delegating public funding, even though they rarely have any comprehension of the projects they finance. Meanwhile, journalists leech off us scientists to fudge sensationalist headlines, and "special interest groups" interfere with academia constantly, even though their very existence conflicts with objective open-mindedness. Worst of all, there's rumors that extralegal forces are manipulating funding, promotions, and even data for nefarious purposes.

On occasion, the parasites have cramped even my boundless zeal. Just a few years back, I had an insight into the workings of the universe-- a fairly common occurrence for me-- and I predicted that the our world might be at risk of a cosmological disaster. When I tried to obtain funding at the Annual Cosmology Conference of Tetra to investigate the possibility, I got no response from political sponsors. They didn't seem excited or even express confusion-- they mentally flatlined. Just days before, \cSciOne{} presented a similar idea at another conference. Yet \cSciOne{\they} received twice the requested resources, while my mentors had to beg for me to be added to the field research team for the project today! I don't understand people, sometimes.

Actually, I don't understand people, ever. People say I'm bad at social skills, and I feel thwarted even when dealing with my so-called peers. They can be so petty, slow and unscientific that I solve derivatives in my head, sometimes, just to distract myself from their fatuity. But I fantasize that, in this alternate universe, I'll meet a parallel version of myself, who doubtless will be the smartest person in their world. Since this inkling draws more on wish fulfillment than legitimate evidence, I haven't told anyone else, but I hope I'm correct. Perhaps I'll finally find someone worthy of my company!

At any rate, this project will be fascinating. \cSciOne{} and I have several official assignments-- to access a parallel world, obtain the energy source that will save our own world, and craft specialized technology in the field, as required by environmental conditions. I'm not sure what the new universe will look like, but it should finally provide me with a chance to prove my intellectual prowess. To this end, I'll of course have to outshine the lead scientist on my team, \cSciOne{}, a task that shouldn't be too difficult. Yet \cSciOne{\they} confuses me more than anyone else does, which is saying something. \cSciOne{\they} doesn't radiate intelligence as I do, yet somehow we both hypothesized this catastrophe. Did \cSciOne{\they} steal my idea? Why did \cSciOne{\they} receive so much warmer a response? I feel like I'm facing a system of equations with one too many variables.

That's it, I've decided. Today, I will do the most impressive work of my career-- I might even play nice with the politicians-- and I will attempt to also unravel the human mysteries of \cSciOne{} and demolish \cSciOne{\their} pedestal, if it turns out undeserved. As of tomorrow, I might be the hero of my world.

\emph{Five minutes after entering the portal:}


Well, something's gone terribly wrong-- the result of entrusting this project to \cSciOne{}, no doubt. Though I'm not positive I could have predicted this mess myself...

Parts of the secret base where we set up our portal appear to have fused with a base in the alternate universe, suggesting a closer link between our worlds than previously theorized. There's a parallel version of our team, standing right in front of us. I have a feeling that saving our own world will prove harder than we thought.
On the bright side, I do indeed have an alternate considered brilliant in \cMageOne{\their} world, though \cMageOne{\they}'s been spouting nonsense about magic for the past few minutes. At least I was correct about something ...

Now, it's time to save the world. All this stupid bureaucracy is getting in my way again. Before I can start work, I have to write a silly proposal and get it approved ... Maybe I can just get \cSciOne{} to do this busy work.

\begin{itemz}[Goals]
  \item Research the parallel world
  \item Write the proposal and get permission from \cPoliOne{} to build the machine. 
  \item Create technology for your world's team
  \item Look into whether you have a double. Maybe you'd finally find someone able to have an intelligent conversation with you. 
  \item Investigate \cSciOne{} and expose any wrongdoing you discover
\end{itemz}

\begin{itemz}[Notes]
	\item You are incredibly narcissistic and difficult to work with. Roleplay accordingly.  
	\item You start the game with 5 gold. 
\end{itemz}


\begin{contacts}
  \contact{\cSciOne{}} The senior scientist on this mission.
  \contact{\cPoliOne{}} The Secretary of Defense, who hopefully won't get in your way.
  \contact{\cTech{}} The technician who recharges your batteries. Might help you outdo \cSciOne{}.
\end{contacts}



\end{document}



