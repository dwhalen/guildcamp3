%%%%%
%%
%% Character sheets live in this directory.  This file doubles as a
%% latex'able example charsheet.
%%
%% _template.tex serves as a bare-bones template suitable for
%% copying when starting a new sheet.
%%
%% Character macros (in ../Lists/char-LIST.tex, presumably) each
%% have a file that lives here.  The argument to \name{...} probably
%% should be the macro for the given character, which will generate
%% the charsheet's name (and print out lists of the characters stuff
%% at the end) as specified in char-LIST.tex.  However, you can also
%% just use \name{Some Text} if you want.
%%
%%%%%

\documentclass[char]{guildcamp3}
\begin{document}

\name{\cSciTwo{}}

%% This sort of use of \updatemacro is covered in
%% Extras/README-namemappings.
\updatemacro{\cNPC}{
  \unknownplayer %% doesn't know what he looks like
  }


%% Using a Char macro with an empty argument, like \cTest{}, will
%% produce the \usual namemapping (see Extras/README-namemappings).
%% Introduce a character with \intro, e.g. \cTest{\intro}, to get a
%% more full name.  \intro can be used whenever it fits into the text
%% flow.

It's not narcissism, just objective fact. I'm the smartest person I've ever met.

I could read sentences at age 1. I entered university at age 15. It's only been ten years since then, but I've been rising speedily through the scientific ranks in this science-focused society. I adore my work! And though I've yet to reach a major breakthrough, I am encouraged and mentored by some of the top members of my profession.

By the scientists, at any rate. The politicians who deal with the scientific world never fail to make me fume. They dominate science by delegating public funding, even though they rarely have any comprehension of the projects they finance. Meanwhile, journalists leech off us scientists to fudge sensationalist headlines, and special interest groups interfere with academia constantly, even though their very existence conflicts with objective open-mindedness. Worst of all, there's rumors that extralegal forces are manipulating funding, promotions, and even data for nefarious purposes.

On occasion, the parasites have cramped even my boundless zeal. Just a few years back, I had an insight into the workings of the universe-- a fairly common occurrence for me-- and I predicted that our world could collide with a parallel world. When I tried to obtain funding at the Annual Cosmology Conference of Tetra to investigate the possibility, I got no response from political sponsors. They didn't seem excited or even express confusion-- they mentally flatlined. Just a few later, \cScientistOne presented a similar idea at another conference. Yet \cScientistOne{\they} received twice the requested resources, and my mentors had to beg for me to be added to the field research team for the convergence today! I don't understand people, sometimes.

Actually, I don't understand people, ever. People say I'm bad at social skills, and I feel thwarted even when dealing with my so-called peers. They can be so petty, slow and unscientific that I solve derivatives in my head, sometimes, just to distract myself from their fatuity. I have a hunch about this convergence, today, that I'll meet a parallel version of myself, who doubtless will be the smartest person in their world. Since this inkling draws more on wish fulfillment than legitimate evidence, I haven't told anyone else, but I hope I'm correct. Perhaps I'll finally find someone worthy of my company.

At any rate, this project will be fascinating. \cScientistOne and I have two official assignments-- to research the parallel world and to craft specialized technology, as required by environmental conditions. I'm not sure what the new universe will look like, but it should finally provide me with a chance to prove my intellectual prowess. To this end, I'll of course have to outshine the lead scientist on my team, \cScientistOne, which isn't much of a task. Honestly, \cScientistOne confuses me more than anyone else does, which is saying something. \cScientistOne{\They} doesn't radiate intelligence as I do, yet we somehow arrived at the same theory of the convergence. Did \cScientistOne{\they} steal my idea? Why did \cScientistOne{\they} receive so much warmer a response? I feel like I'm facing a system of equations with one too many variables.

That's it, I've decided. Today, I will do the most impressive work of my career-- I might even play nice with the politicians-- and I will attempt to also unravel the human mysteries of \cScientistOne and demolish \cScientistOne{\their} pedestal, if it turns out undeserved. As of tomorrow, I might be the hero of my world.

\begin{itemz}[Goals]
  \item Things to do
  \item Research the parallel world
  \item Create technology for your world's team
  \item Look into whether you have a double
  \item Investigate \cScientistOne and expose any wrongdoing you discover
\end{itemz}


%%%%%
%% List contacts, using \contact{<char macro>}
\begin{contacts}
  \contact{\cScientistOne{}} The senior scientist on this mission.
  \contact{\cPoliticianOne{}} The Secretary of Defense, who hopefully won't get in your way.
  \contact{\cPoliticianTwo{}} The Undersecretary of Technology, who will oversee your work. Might cause trouble.
  \contact{\cSpecOpsOne{}} Someone to do with security. Might help you obtain information or resources.
  \contact{\cSpecOpsTwo{}} Someone else related to security. Might help you obtain information or resources.
  \contact{\cTechnician{}} The technician who recharges your batteries. Might help you outdo \cSciOne.
\end{contacts}


%%%%%
%% \starttag{<tag>} <elements> \endtag 
%% Valid <tag> values are blues, greens, abils, combat, mems, items,
%% whites, notebooks, cash, signs, ids.  These each correspond to a
%% type of macro defined in Lists/.
%%
%% By using \starttag, you can give this character <elements> of the
%% type corresponding to <tag>.
%%
%% Multiple uses of the same <tag> will simply add together.
\starttag{mems}

  \mTest{}

  \memfold{``Rosebud''}{Rosebud!  That was the name of\ldots the name
  of\ldots darn, you forget.}

  \startmembook{Book of Mempackets}

    \mempage{if you see something blue}{Hey, that's blue!  Oh, you
    remember, blue is your favorite color.  You really like blue
    things, especially blue tentacles.  You wonder why\ldots}

    \mempage{``Octy''}{Octy!  You remember Octy now!  She was your pet
    blue octopus when you were a young child living offshore.  Oh, the
    fun times you had!

    You used to go swimming and diving with Octy all the time.  This
    was years ago.  What happened?  You still can't remember\ldots but
    you know you haven't even thought of her since you were small.}

  \endmembook

\endtag

\starttag{abils}
  \ability{Amazing Powers}{You can do strange and amazing things.}{I
  do something strange and amazing.}
\endtag


\end{document}
