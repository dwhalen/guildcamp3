%%%%%
%%
%% Character sheets live in this directory.  This file doubles as a
%% latex'able example charsheet.
%%
%% _template.tex serves as a bare-bones template suitable for
%% copying when starting a new sheet.
%%
%% Character macros (in ../Lists/char-LIST.tex, presumably) each
%% have a file that lives here.  The argument to \name{...} probably
%% should be the macro for the given character, which will generate
%% the charsheet's name (and print out lists of the characters stuff
%% at the end) as specified in char-LIST.tex.  However, you can also
%% just use \name{Some Text} if you want.
%%
%%%%%

\documentclass[char]{guildcamp3}
\begin{document}

\name{\cSciOne{}}


\updatemacro{\cNPC}{
  \unknownplayer %% doesn't know what he looks like
  }

How did it get so far?

You were approached thirteen years ago, when you were still in university, by a politician who obtained a grant for you by pushing strongly. Too strongly. You were planning to explore commercial uses for Compound 9, which showed promise as a lightweight alternative to cement. Little did you know that Compound 9 was already making the rounds in the shadowy underworld. Dealers were touting it as a potent hallucinogen, while mob bosses wanted the government to sanction its creation for legal purposes, in order to secure a steady supply. They wanted your research to legitimize Compound 9, and they'd provide you funding and connections in return. You were fine with receiving help from the mob-connected politician-- after all, the scientific world is highly competitive, and, as a novice researcher, you were grateful for the leg-up. Unfortunately, your research showed that Compound 9 couldn't actually function as a building material. The mob forced you to fudge your results, and that's where the troubles started.

The mob has so much dirt on you that you're practically their slave. You research what they tell you research, you report the results that they want. You fundamentally understand science, though, and you've veiled your projects with enough hedging and "confounding factors" that other academics haven't caught on.


%% Using a Char macro with an empty argument, like \cTest{}, will
%% produce the \usual namemapping (see Extras/README-namemappings).
%% Introduce a character with \intro, e.g. \cTest{\intro}, to get a
%% more full name.  \intro can be used whenever it fits into the text
%% flow.
Your friend from out of town, \cSomeGuy{\intro}, called you up one
morning about a week ago.  \cSomeGuy{} told you to meet up with this
person named \cNPC{}, whom you've never met.  \cNPC{}, whose full name
is \cNPC{\intro}, is supposed to be really awesome and have a package
for you.

%% You can skip the advanced namemapping commands, and instead use
%% \full, \fullplain (full name without prefixes or suffixes),
%% \formal, and \informal.  You can also nest these inside identities,
%% such as \nick{} (see Extras/README-identity).
That bears repeating: \cNPC{\nick{\informal}}, whose full name is
\cNPC{\full}, is supposed to be really awesome and have a package for
you.

%% For pronouns and other gender-dependent words, you can use the
%% pronoun commands defined in Lists/char-LIST.tex to automatically
%% control them based on the character's gender.  For example,
%% \cTest{\They} will produce He, She, It, or He/She, based on
%% \cTest's \MYsex field.  You can define your own pronouns in
%% Lists/char-LIST.tex, as well.
\cSomeGuy{} is a pretty good friend.  \cSomeGuy{\They} used to be your
college roommate.  When \cSomeGuy{\they} called you up, you were
pleasantly surprised to hear from \cSomeGuy{\them}.

%% \me{} produces whatever the argument to \name{...} is.  If, like
%% usual, the argument was a char Macro, \me{} is an alias for that
%% macro.
Lots of people think your name, \me{\intro}, is funny.  You're not
sure why; you think it's a fine name.



%%%%%
%% The itemz environment is a list environment similar to itemize.
%% The typesetting is very tight, and matches that used by the lists
%% at the end of character sheets.  It takes an optional argument that
%% acts as a title for the list.  The enum environment is a similar
%% variation of the enumerate environment, and the desc environment is
%% similar to description.
\begin{itemz}[Goals]
  \item Things to do
  \item Governments to topple
  \item Worlds to dominate
\end{itemz}

\begin{itemz}[Notes]
  \item You were born in London.
  \item You went to MIT, and never left.
\end{itemz}


%%%%%
%% List contacts, using \contact{<char macro>}
\begin{contacts}
  \contact{\cNPC{}} This person you've never met.
  \contact{\cSomeGuy{}} Your friend from out of town.
\end{contacts}


%%%%%
%% \starttag{<tag>} <elements> \endtag 
%% Valid <tag> values are blues, greens, abils, combat, mems, items,
%% whites, notebooks, cash, signs, ids.  These each correspond to a
%% type of macro defined in Lists/.
%%
%% By using \starttag, you can give this character <elements> of the
%% type corresponding to <tag>.
%%
%% Multiple uses of the same <tag> will simply add together.
\starttag{mems}

  \mTest{}

  \memfold{``Rosebud''}{Rosebud!  That was the name of\ldots the name
  of\ldots darn, you forget.}

  \startmembook{Book of Mempackets}

    \mempage{if you see something blue}{Hey, that's blue!  Oh, you
    remember, blue is your favorite color.  You really like blue
    things, especially blue tentacles.  You wonder why\ldots}

    \mempage{``Octy''}{Octy!  You remember Octy now!  She was your pet
    blue octopus when you were a young child living offshore.  Oh, the
    fun times you had!

    You used to go swimming and diving with Octy all the time.  This
    was years ago.  What happened?  You still can't remember\ldots but
    you know you haven't even thought of her since you were small.}

  \endmembook

\endtag

\starttag{abils}
  \ability{Amazing Powers}{You can do strange and amazing things.}{I
  do something strange and amazing.}
\endtag


\end{document}
