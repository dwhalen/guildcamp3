%%%%%
%%
%% Character sheets live in this directory.  This file doubles as a
%% latex'able example charsheet.
%%
%% _template.tex serves as a bare-bones template suitable for
%% copying when starting a new sheet.
%%
%% Character macros (in ../Lists/char-LIST.tex, presumably) each
%% have a file that lives here.  The argument to \name{...} probably
%% should be the macro for the given character, which will generate
%% the charsheet's name (and print out lists of the characters stuff
%% at the end) as specified in char-LIST.tex.  However, you can also
%% just use \name{Some Text} if you want.
%%
%%%%%

\documentclass[char]{guildcamp3}
\begin{document}

\name{\cMageTwo{}}

%% This sort of use of \updatemacro is covered in
%% Extras/README-namemappings.
\updatemacro{\cNPC}{
  \unknownplayer %% doesn't know what he looks like
  }

%% quote examples
%\bigquote{``Use this macro for large quotes of prose and such.  It
%justifies everything like a paragraph, except with no
%indentation.''}{-- The Author}

%\cenquote{``This macro is good\\ For shorter quotes\\ Or things like
%song lyrics:\\ It centers.''}{-- The Author}


%% \TODO outputs to the page and the terminal.  It is used for
%% reminders of future work, and a convenient way to build a short
%% outline for a sheet in-progress.
%\TODO{This is a test character sheet.}

%% Using a Char macro with an empty argument, like \cTest{}, will
%% produce the \usual namemapping (see Extras/README-namemappings).
%% Introduce a character with \intro, e.g. \cTest{\intro}, to get a
%% more full name.  \intro can be used whenever it fits into the text
%% flow.
%Your friend from out of town, \cSomeGuy{\intro}, called you up one
%morning about a week ago.  \cSomeGuy{} told you to meet up with this
%person named \cNPC{}, whom you've never met.  \cNPC{}, whose full name
%is \cNPC{\intro}, is supposed to be really awesome and have a package
%for you.

%% You can skip the advanced namemapping commands, and instead use
%% \full, \fullplain (full name without prefixes or suffixes),
%% \formal, and \informal.  You can also nest these inside identities,
%% such as \nick{} (see Extras/README-identity).
%That bears repeating: \cNPC{\nick{\informal}}, whose full name is
%\cNPC{\full}, is supposed to be really awesome and have a package for
%you.

%% For pronouns and other gender-dependent words, you can use the
%% pronoun commands defined in Lists/char-LIST.tex to automatically
%% control them based on the character's gender.  For example,
%% \cTest{\They} will produce He, She, It, or He/She, based on
%% \cTest's \MYsex field.  You can define your own pronouns in
%% Lists/char-LIST.tex, as well.
%\cSomeGuy{} is a pretty good friend.  \cSomeGuy{\They} used to be your
%college roommate.  When \cSomeGuy{\they} called you up, you were
%pleasantly surprised to hear from \cSomeGuy{\them}.

%% \me{} produces whatever the argument to \name{...} is.  If, like
%% usual, the argument was a char Macro, \me{} is an alias for that
%% macro.
%Lots of people think your name, \me{\intro}, is funny.  You're not
%sure why; you think it's a fine name.

You are a mage. This is something that you have earned through long hours, years of study, sacrifice and work. In a world where even the least have the ability to manipulate magic, it take something very special to become a mage. You succeeded despite the world being against you the odds stacked in favor of the house, dealt a bad hand you have managed to drag yourself up into being one of the top researchers of  \bMagicWorld{}. You lost both of your parents early in life, one to the void, and one simply walking out on you. Hard times in your youth did not prevent you from displaying great talent in the manipulation of the fabric of reality. This lead to you being chosen for arcane study, paid by scholarship you devoted yourself to proving that you will be the best. Now you find yourself working on a classified base delving into the mysteries of reality. Though throughout your studies you found one person, \cMageOne{}, to always be right there with \cMageOne{\Their} own discovery, \cMageOne{\Their} own latest master craft. Obviously a spoiled noble brat you has never had to work just to survive you want to beat \cMageOne{\Them} in the way that really matters, publicly, and irrefutably. The best way to do this is to get your name on the department publishing board used to record the achievements of the mages.

Recently you have come to know a dark secret that is befalling your world, the weave is becoming unstable which would lead to the end of your world. Thankfully your department has been able to detect a stabilizing force in another place, a place outside this world, outside of conventional travel. While many have thought of alternate universes being out there, you have helped to create a method of taking that theory and making it a reality. There are somethings on the otherside that should be of use for stabilizing the weave, at least that is what your rituals have been able to divine. Small transient portals have been able to be opened, enough to see that this other world is not inhospitable, the air is breathable, though you have seen glimpses of the oddest sort of buildings, a shiny grey and glass. You wonder what forms of magic there are in this land, perhaps different techniques are used on the other side, so exciting this is. Your rituals have narrowed down the location of the artifacts you need, and your overseers have decided that a massive form of the ritual must be enacted. Under the watch of your superiors \cNobleOne{\intro} and \cNobleTwo{\intro}, as well as base security, the ritual was put into motion, with the rather unwanted help of \cMageTwo{}. There was a flash a sense of wrongness and a taste of red, there are a number of other people in strange attire, one looks so strangely like your twin, and the walls, they are no longer just the simple stone you are used to but a mix of stone and metal, and your ritual station now has an odd metal mirror of itself on the other side of the room. The ritual seems to have taken the your base and blended it with a base on the otherside. The world is one parallel to yours to the point that there are even doubles of yourselves there. It seems their side mirrors your situation and their attempted to come to your side seeking three artifacts that you know are on base somewhere. You and your rival will need to work with these newcomers to create something to analyze the artifacts in question to see how they can stabilize the worlds. Of course any plans will need to be signed off on by \cNobleTwo{} before any building is allowed, bureaucracy at its finest. Also fortunate is that while the rest of bothe worlds are cut off at the moment \cPaladin{} is here, meaning your mana will be able to be recharged, thank the gods.

On the thought of the doubles, you recognize \cRogueTwo{} as your \cRogueTwo{\parent} so it is quite possible that they have a double as well, which mean you have a second chance to have a \cRogueTwo{\parent}, one that is not a worthless walkout. 





%%%%%
%% The itemz environment is a list environment similar to itemize.
%% The typesetting is very tight, and matches that used by the lists
%% at the end of character sheets.  It takes an optional argument that
%% acts as a title for the list.  The enum environment is a similar
%% variation of the enumerate environment, and the desc environment is
%% similar to description.
\begin{itemz}[Goals]
  \item Figure out how to save the world
  \item Win the research race against \cMageTwo{}
  \item propose funding and get signed off by \cNobleTwo{}
  \item Figure out who your parents double is and get to know them.
  \item look for evidence of magic in the tech world
  \item Create useful items with your arcane knowledge
\end{itemz}

\begin{itemz}[Notes]
  \item You were born in London.
  \item You went to MIT, and never left.
\end{itemz}


%%%%%
%% List contacts, using \contact{<char macro>}
\begin{contacts}
  \contact{\cMageTwo{}} A child rival that is still around.
  \contact{\cNobleOne{}} The Secretariat of Defensive Measures, the ranking noble on the base
  \contact{\cNobleTwo{}} The Junior Secretariat of Arcane Research
  \contact{\cPaladin{}} The Paladin for the base, very well respected and source of mana 
  \contact{\cRogueOne{}} One of the security personnel for the base
  \contact{\cRogueTwo{}} One of the security personnel for the base
  \contact{\cServant{}} The loyal servant of /cNobleOne{} 
\end{contacts}


%%%%%
%% \starttag{<tag>} <elements> \endtag 
%% Valid <tag> values are blues, greens, abils, combat, mems, items,
%% whites, notebooks, cash, signs, ids.  These each correspond to a
%% type of macro defined in Lists/.
%%
%% By using \starttag, you can give this character <elements> of the
%% type corresponding to <tag>.
%%
%% Multiple uses of the same <tag> will simply add together.
\starttag{mems}

  \mTest{}

  \memfold{``Rosebud''}{Rosebud!  That was the name of\ldots the name
  of\ldots darn, you forget.}

  \startmembook{Book of Mempackets}

    \mempage{if you see something blue}{Hey, that's blue!  Oh, you
    remember, blue is your favorite color.  You really like blue
    things, especially blue tentacles.  You wonder why\ldots}

    \mempage{``Octy''}{Octy!  You remember Octy now!  She was your pet
    blue octopus when you were a young child living offshore.  Oh, the
    fun times you had!

    You used to go swimming and diving with Octy all the time.  This
    was years ago.  What happened?  You still can't remember\ldots but
    you know you haven't even thought of her since you were small.}

  \endmembook

\endtag

\starttag{abils}
  \ability{Amazing Powers}{You can do strange and amazing things.}{I
  do something strange and amazing.}
\endtag


\end{document}
