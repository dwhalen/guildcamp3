%%%%%
%%
%% Character sheets live in this directory.  This file doubles as a
%% latex'able example charsheet.
%%
%% _template.tex serves as a bare-bones template suitable for
%% copying when starting a new sheet.
%%
%% Character macros (in ../Lists/char-LIST.tex, presumably) each
%% have a file that lives here.  The argument to \name{...} probably
%% should be the macro for the given character, which will generate
%% the charsheet's name (and print out lists of the characters stuff
%% at the end) as specified in char-LIST.tex.  However, you can also
%% just use \name{Some Text} if you want.
%%
%%%%%

\documentclass[char]{guildcamp3}
\begin{document}

\name{\cSpecOpTwo{}}

%% This sort of use of \updatemacro is covered in
%% Extras/README-namemappings.
\updatemacro{\cNPC}{
  \unknownplayer %% doesn't know what he looks like
  }

%% quote examples
\bigquote{``Use this macro for large quotes of prose and such.  It
justifies everything like a paragraph, except with no
indentation.''}{-- The Author}

\cenquote{``This macro is good\\ For shorter quotes\\ Or things like
song lyrics:\\ It centers.''}{-- The Author}


%% \TODO outputs to the page and the terminal.  It is used for
%% reminders of future work, and a convenient way to build a short
%% outline for a sheet in-progress.
\TODO{Write spec op two's character sheet.}

You couldn't be more honored to have been selected to be on the defense team for this expedition to save the world. Coming from a poor family, you never dreamed of such an honor. Of course, you didn't exactly expect it to work out this way, with discovering an entire parallel universe in which there apparently exist doubles of everyone. Upon this portal opening, you just saw a bunch of people - who all seemed to be somehow similar to those you brought. Strangely enough, you seem to be the double of the Magical World's religious leader, \cPaladin{\intro}. You were never a religious \pronoun{\human} - never even went to church as a child. So who really knows what's up with that.

Now, despite the 

%% Using a Char macro with an empty argument, like \cTest{}, will
%% produce the \usual namemapping (see Extras/README-namemappings).
%% Introduce a character with \intro, e.g. \cTest{\intro}, to get a
%% more full name.  \intro can be used whenever it fits into the text
%% flow.

%%%%%
%% The itemz environment is a list environment similar to itemize.
%% The typesetting is very tight, and matches that used by the lists
%% at the end of character sheets.  It takes an optional argument that
%% acts as a title for the list.  The enum environment is a similar
%% variation of the enumerate environment, and the desc environment is
%% similar to description.
\begin{itemz}[Goals]
  \item Ensure the safety of the politicians, and the technical world in general.
  \item Inventory the remaining supplies and ensure the resources needed to save the world are present.
  \item Find a cure for your trances.
  \item Ensure no one finds out about your trances.
  \item Make the mob exchange with \cServant{\informal}.
  \item Make the mob drug exchange with the whole group.
  \item Get enough money to send back to your family at home. 
\end{itemz}

\begin{itemz}[Notes]
  \item You were born in London.
  \item You went to MIT, and never left.
\end{itemz}


%%%%%
%% List contacts, using \contact{<char macro>}
\begin{contacts}
  \contact{\cSpecOpOne{}} Your boss.
  \contact{\cPoliOne{}} The Secretary of Defense. \cPoliOne{they}'s in charge today and you should protect him.
  \contact{\cPoliTwo{}} The Undersecretary of Technology. Protect him too.
  \contact{\cServant{}} The corresponding mobster you must exchange things with for the special mission. Be sure you do this or the mob will have your head.
  \contact{\cPaladin{}} The religious leader of the Magical World. For some reason you're \cPaladin{}'s double. 
\end{contacts}


%%%%%
%% \starttag{<tag>} <elements> \endtag 
%% Valid <tag> values are blues, greens, abils, combat, mems, items,
%% whites, notebooks, cash, signs, ids.  These each correspond to a
%% type of macro defined in Lists/.
%%
%% By using \starttag, you can give this character <elements> of the
%% type corresponding to <tag>.
%%
%% Multiple uses of the same <tag> will simply add together.
\starttag{mems}

  \mTest{}

  \memfold{``Rosebud''}{Rosebud!  That was the name of\ldots the name
  of\ldots darn, you forget.}

  \startmembook{Book of Mempackets}

    \mempage{if you see something blue}{Hey, that's blue!  Oh, you
    remember, blue is your favorite color.  You really like blue
    things, especially blue tentacles.  You wonder why\ldots}

    \mempage{``Octy''}{Octy!  You remember Octy now!  She was your pet
    blue octopus when you were a young child living offshore.  Oh, the
    fun times you had!

    You used to go swimming and diving with Octy all the time.  This
    was years ago.  What happened?  You still can't remember\ldots but
    you know you haven't even thought of her since you were small.}

  \endmembook

\endtag

\starttag{abils}
  \ability{Amazing Powers}{You can do strange and amazing things.}{I
  do something strange and amazing.}
\endtag


\end{document}
