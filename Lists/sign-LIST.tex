%%%%%
%%
%% This file sets up the Sign and Label datatypes and creates Sign and
%% Label macros.
%%
%% Signs generally represent interesting parts of game area, usually
%% as things posted on walls.  Labels represent other things, often on
%% or inside envelopes, that are part of complex mechanics.
%%
%% The default value for \MYloc will inherit location from the Place
%% or Sign most immediately up the ownership tree.  Override this by
%% setting \MYloc to anything (even blank).
%%
%% Sign is for full-sized signs that would cover most of a large
%% manila envelope; SignMedium is for signs sized to half-sized manila
%% envelopes; SignSmall is for signs sized for small manila envelopes
%% (the same size as item cards).  Label, LabelMedium, and LabelSmall
%% are analagous, but they don't have a \takedownby note at the
%% bottom.  You can always use a sign or label without an envelope or
%% with a differently-sized envelope.  Choose which based on
%% visibility and content.
%%
%% SignTiny is for signs you want to be hard to find; it is small and
%% does not have a \takedownby note.  SignDot is for a very small
%% "dot" which only has a title.
%%
%% SignStrip produces a strip of paper (without a \takedownby note)
%% with labels on the outside that show on both sides if you fold it
%% in half.  These are a convenient alternative to sub-envelopes. They
%% can also be used for "s-packets" taped to walls (see
%% Extras/README-s-packets).
%%
%% LabelCover produces a label similar to the cover to a research
%% notebook.  LabelPage, likewise, produces a page.
%%
%% EOG is for full-sized end-of-game signs.
%%
%%%%%

\DECLARESUBTYPE{Sign}{Element}
\PRESETS{Sign}{
  \FD\MYloc	{\mylocation} %% real-space location
  \FD\MYtext	{} %% text of sign
  }
\POSTSETS{Sign}{
  \edef\mylocation{\MYloc}
  \protected@edef\@ownerstring{%
    \MYname%
    \ifx\mylocation\empty\else\ (\mylocation)\fi%
    }
  }
\def\mylocation{}

\def\loc#1{\rs\MYloc{#1}}

\DECLARESUBTYPE{SignMedium}{Sign}
\DECLARESUBTYPE{SignSmall}{Sign}
\DECLARESUBTYPE{SignTiny}{Sign}
\DECLARESUBTYPE{SignDot}{Sign}
\PRESETS{SignDot}{\s\MYtext{}}

\DECLARESUBTYPE{Label}{Sign}
\PRESETS{Label}{\s\MYloc{}}
\DECLARESUBTYPE{LabelMedium}{Label}
\DECLARESUBTYPE{LabelSmall}{Label}

\DECLARESUBTYPE{SignStrip}{Sign}
\DECLARESUBTYPE{LabelCover}{Label}
\DECLARESUBTYPE{LabelPage}{Label}

\DECLARESUBTYPE{EOG}{Sign}
\PRESETS{EOG}{%
  \s\MYname	{End Of Game}
  \s\MYtext	{{\bf\Huge You may not pass through here.}}
  }


%%%%%
%% \signbig[<location>]{<name>}{<text>}
%% \eog[<location>]
%%
%% \signmdeium[<location>]{<name>}{<text>}
%% \signsmall[<location>]{<name>}{<text>}
%% \signtiny[<location>]{<name>}{<text>}
%% \signdot[<location>]{<name>}
%%
%% \labelbig{<name>}{<text>}
%% \labelmedium{<name>}{<text>}
%% \labelsmall{<name>}{<text>}
%%
%% \signstrip[<location>]{<name>}{<text>}
%% \labelcover{<name>}{<text>}
%% \labelpage{<name>}{<text>}
\newinstance{Sign}{\signbig[3][\mylocation]}{
  \s\MYloc{#1}\s\MYname{#2}\s\MYtext{#3}}
\newinstance{EOG}{\eog[1][\mylocation]}{\s\MYloc{#1}}

\newinstance{SignMedium}{\signmedium[3][\mylocation]}{
  \s\MYloc{#1}\s\MYname{#2}\s\MYtext{#3}}
\newinstance{SignSmall}{\signsmall[3][\mylocation]}{
  \s\MYloc{#1}\s\MYname{#2}\s\MYtext{#3}}
\newinstance{SignTiny}{\signtiny[3][\mylocation]}{
  \s\MYloc{#1}\s\MYname{#2}\s\MYtext{#3}}
\newinstance{SignDot}{\signdot[2][\mylocation]}{
  \s\MYloc{#1}\s\MYname{#2}}

\newinstance{Label}{\labelbig[2]}{
  \s\MYname{#1}\s\MYtext{#2}}
\newinstance{LabelMedium}{\labelmedium[2]}{
  \s\MYname{#1}\s\MYtext{#2}}
\newinstance{LabelSmall}{\labelsmall[2]}{
  \s\MYname{#1}\s\MYtext{#2}}

\newinstance{SignStrip}{\signstrip[3][\mylocation]}{
  \s\MYloc{#1}\s\MYname{#2}\s\MYtext{#3}}
\newinstance{LabelCover}{\labelcover[2]}{
  \s\MYname{#1}\s\MYtext{#2}}
\newinstance{LabelPage}{\labelpage[2]}{
  \s\MYname{#1}\s\MYtext{#2}}


%%%%%
%% \sEOG{}
%% use \sEOg[\loc{<location>}]{} for EOG sign at a specific place
\NEW{EOG}{\sEOG}{
  }


%%%%%%%%%%%%%%%%%%%%%%%%%%%%%%%%%%%%%%%%%%%%%%%%%%%%%%%%%%%%%%%%%%

\NEW{Sign}{\sTest}{
  \s\MYname	{A Room}
  \s\MYloc	{10-250}
  \s\MYtext	{A lecture hall with large, sliding blackboards.}
  }
  
\NEW{Sign}{\sArmory}{
  \s\MYname {Armory}
  \s\MYloc	{200-030}
  \s\MYtext {This door cannot remain open for more than 15 seconds. Unless you know otherwise, you must override the security system to open this door. In order to override the system, four people must simultaneously touch this sign for a period of 30 seconds. This is an interruptible mechanic.}
  }

\NEW{Sign}{\sMageWorkbench}{
	\s\MYname {The Mages' Ritual Desk}
	\s\MYloc	{200-002}
	\s\MYtext {A desk covered with all sorts of magical items. There are also puzzles here.}
	\s\MYwhites{\wMagePuzzleOne{}\wMagePuzzleTwo{}\wMagePuzzleThree{}\wMagePuzzleFour{}\wMagePuzzleFive{}}
	}

\NEW{Sign}{\sSciWorkbench}{
	\s\MYname {The Scientists' Workbench}
	\s\MYloc	{200-002}
	\s\MYtext {A workbench covered in all sorts of mechatronic contraptions. There are also puzzles here.}
	\s\MYwhites{\wSciPuzzleOne{}\wSciPuzzleTwo{}\wSciPuzzleThree{}\wSciPuzzleFour{}\wSciPuzzleFive{}}
}

\NEW{Sign}{\sBarracks}{
	\s\MYname {Barracks}
	\s\MYloc	{200-034}
	\s\MYtext {Headquarters for base guards.}
}

\NEW{Sign}{\sLockers}{
	\s\MYname {Lockers}
	\s\MYloc	{By the stairs}
	\s\MYtext {Assorted personnel lockers.}
}

\NEW{Sign}{\sAltar}{
	\s\MYname {Altar}
	\s\MYloc	{200-032}
	\s\MYtext {An altar to the gods. It is inscribed with arcane runes. A lit candle burns here.}
}

\NEW{Sign}{\sTechWorkbench}{
	\s\MYname {Technician's Workbench}
	\s\MYloc	{200-032}
	\s\MYtext {A workbench covered in bits of technological components and tools.}
}

\NEW{Sign}{\sConference}{
	\s\MYname {Conference Room}
	\s\MYloc	{200-013}
	\s\MYtext {A lavishly decorated conference room. The walls seem to shift from thick stone blocks to polished wood depending on what part of the room they are in.}
}

\NEW{Sign}{\sGenStorage}{
	\s\MYname {General Storage}
	\s\MYloc	{Hall by 002}
	\s\MYtext {A general storage area filled with boxes and assorted items.}
}

\NEW{Sign}{\sMStorage}{
	\s\MYname {Mages' Storage}
	\s\MYloc	{Hall by 034}
	\s\MYtext {Storage area filled with arcane tools and supplies.}
}

\NEW{Sign}{\sSStorage}{
	\s\MYname {Scientists' Storage}
	\s\MYloc	{200-015}
	\s\MYtext {Storage area filled with assorted tools and supplies.}
}

\NEW{Sign}{\sSciOneLocker}{
	\s\MYname	{Locker}
	\s\MYloc	{}
	\s\MYtext	{Only open if you are \cSciOne{} unless you know otherwise. This locker can be lock-picked (medium). You can break into it with a combined CR of 9. Bulky items may not be added.}
}

\NEW{Sign}{\sSciTwoLocker}{
	\s\MYname	{Locker}
	\s\MYloc	{}
	\s\MYtext	{Only open if you are \cSciTwo{} unless you know otherwise. This locker can be lock-picked (medium). You can break into it with a combined CR of 9. Bulky items may not be added.}
}

\NEW{Sign}{\sMageOneLocker}{
	\s\MYname	{Locker}
	\s\MYloc	{}
	\s\MYtext	{Only open if you are \cMageOne{} unless you know otherwise. This locker can be lock-picked (medium). You can break into it with a combined CR of 9. Bulky items may not be added.}
}

\NEW{Sign}{\sMageTwoLocker}{
	\s\MYname	{Locker}
	\s\MYloc	{}
	\s\MYtext	{Only open if you are \cMageTwo{} unless you know otherwise. This locker can be lock-picked (medium). You can break into it with a combined CR of 9. Bulky items may not be added.}
}

\NEW{Sign}{\sNobleOneLocker}{
	\s\MYname	{Locker}
	\s\MYloc	{}
	\s\MYtext	{Only open if you are \cNobleOne{} unless you know otherwise. This locker can be lock-picked (hard). You can break into it with a combined CR of 9. Bulky items may not be added.}
}



\NEW{Sign}{\sPolOneLocker}{
	\s\MYname	{Locker}
	\s\MYloc	{}
	\s\MYtext	{Only open if you are \cPoliOne{} unless you know otherwise. This locker can be lock-picked (hard). You can break into it with a combined CR of 9. Bulky items may not be added.}
}

\NEW{Sign}{\sPolTwoLocker}{
	\s\MYname	{Locker}
	\s\MYloc	{}
	\s\MYtext	{Only open if you are \cPoliTwo{} unless you know otherwise. This locker can be lock-picked (hard). You can break into it with a combined CR of 9. Bulky items may not be added.}
}

\NEW{Sign}{\sRogueOneLocker}{
	\s\MYname	{Locker}
	\s\MYloc	{}
	\s\MYtext	{Only open if you are \cRogueOne{} unless you know otherwise. This locker can be lock-picked (medium). You can break into it with a combined CR of 9. Bulky items may not be added.}
}

\NEW{Sign}{\sRogueTwoLocker}{
	\s\MYname	{Locker}
	\s\MYloc	{}
	\s\MYtext	{Only open if you are \cRogueTwo{} unless you know otherwise. This locker can be lock-picked (medium). You can break into it with a combined CR of 9. Bulky items may not be added.}
}

\NEW{Sign}{\sSpecOneLocker}{
	\s\MYname	{Locker}
	\s\MYloc	{}
	\s\MYtext	{Only open if you are \cSpecOpOne{} unless you know otherwise. This locker can be lock-picked (medium). You can break into it with a combined CR of 9. Bulky items may not be added.}
}

\NEW{Sign}{\sSpecTwoLocker}{
	\s\MYname	{Locker}
	\s\MYloc	{}
	\s\MYtext	{Only open if you are \cSpecOpTwo{} unless you know otherwise. This locker can be lock-picked (medium). You can break into it with a combined CR of 9. Bulky items may not be added.}
}

\NEW{Sign}{\sPaladinLocker}{
	\s\MYname	{Locker}
	\s\MYloc	{}
	\s\MYtext	{Only open if you are \cPaladin{} unless you know otherwise. This locker can be lock-picked (easy). You can break into it with a combined CR of 9. Bulky items may not be added.}
}

\NEW{Sign}{\sTechLocker}{
	\s\MYname	{Locker}
	\s\MYloc	{}
	\s\MYtext	{Only open if you are \cTech{} unless you know otherwise. This locker can be lock-picked (medium). You can break into it with a combined CR of 9. Bulky items may not be added.}
}

\NEW{Sign}{\sServantLocker}{
	\s\MYname	{Locker}
	\s\MYloc	{}
	\s\MYtext	{Only open if you are \cServant{} unless you know otherwise. This locker can be lock-picked (easy). You can break into it with a combined CR of 9. Bulky items may not be added.}
}


\NEW{Sign}{\sGFG}{
	\s\MYname	{The GFG}
	\s\MYloc	{}
	\s\MYtext	{The ultimate machine. The GFG. It's big and fancy and has a few input slots and a gigantic red button and a gigantic yellow lever. How could it possibly function? What goes into these slots? What does the button do? What about the lever?}
}

%%%%%%%%%%%%%%%%%%%%%%%%%%%%%%%%%%%%%%%%%%%%%%%%%%%%%%%%%%%%%%%%%%
